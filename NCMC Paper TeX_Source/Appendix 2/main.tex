\documentclass{article}

% Packages
\usepackage[margin=1in]{geometry}
\usepackage{amsmath,amssymb}
\usepackage{titlesec}
\usepackage{titling}
\usepackage{mathtools}

% Custom unnumbered section format
\titleformat{\section}[block]
  {\normalfont\Large\bfseries}{}{0pt}{}  % Remove section number

% Left-aligned large title
\pretitle{\begin{flushleft}\LARGE\bfseries}
\posttitle{\end{flushleft}}
\preauthor{\begin{flushleft} \large}
\postauthor{\end{flushleft}}
\predate{\begin{flushleft} \large}
\postdate{\end{flushleft}}
\date{}

% Title info
\title{Appendix 2}
\begin{document}

\maketitle

\section{The Null Geodesic}

The path that a light beam, carrying negligible energy, follows 
in the curved spacetime of General Relativity is known as the 
null geodesic. It can be shown from the principle of 
equivalence that the null geodesic follows the form [32]

\[
\frac{\partial^2 x^\lambda}{\partial \sigma^2} + 
\frac{1}{2} g^{\lambda\alpha} \left( \frac{\partial g_{\nu\alpha}}{\partial x^\mu}
+\frac{\partial g_{\beta\mu}}{\partial x^\nu}
-\frac{\partial g_{\mu\nu}}{\partial x^\beta}\right)
\frac{\partial x^\mu}{\partial \sigma}
\frac{\partial x^\nu}{\partial \sigma}
 = 0,
\]
where $x(\sigma)$ represents the trajectory of the light beam parameterized 
by $\sigma$, and $g$ is the spacetime metric at location $x$. 
Recall from the main paper that $ g^{\lambda\alpha}$ denotes the 
inverse metric. The principle of equivalence states that: for any point 
in space and any moment in time, we can always define a free-falling 
inertial frame of reference that the effect of gravity is locally absent. 
In such case, physical laws in this local region can be described by 
only the theory of Special Relativity, thus allowing us to analyze 
infinitesimal motions, including that of light, as inertial motions 
(i.e., motions of object in the absence of external forces) in 
flat spacetime. If we choose the flow of time in these local reference 
frames as our parameterization $\sigma$, and let $x^\alpha$ denote 
the coordinate position of the light beam at a moment in time, then 
the following equality must hold:
\[
\frac{\partial^2 x^\alpha}{\partial \sigma^2} = 0
\]
To constrain the speed of this beam, we shall impose
\[
\eta_{AB} \frac{\partial x^A}{\partial \sigma} \frac{\partial x^B}{\partial \sigma}= 0,
\]
where $\eta_{AB}$ represents the Minkowski metric
\[
\eta_{AB} =
\begin{bmatrix}
-1 & 0 & 0 & 0 \\
0 & 1 & 0 & 0 \\
0 & 0 & 1 & 0 \\
0 & 0 & 0 & 1
\end{bmatrix}
\]
This ensures that
the light beam's trajectory through spacetime 
stays on the null cone, as per the description of Special Relativity.
\\

Now imagine that the light departs this local inertial frame as it travels 
forward. Meanwhile, there exist a varying gravitational field at every 
point in space. Then the light will enter an adjacent region of a different 
spacetime curvature. Since the new region can also be approximated by laws 
of Special Relativity locally, we can setup a new inertial reference frame 
where its metric is simply a linear transformation of $\eta_{AB}$ (i.e., 
the product with Jacobian and its transpose, as the variation in space 
is thought as infinitesimal). This would hold as long as the change of 
spacetime curvature is continuous (e.g., no singularity) along the path 
of the light beam. Furthermore, we can see the new inertial frame as the 
consequence of a coordinate transform acting on the old, thus: 
\[
\frac{\partial \xi^\alpha}{\partial \sigma} = 
\frac{\partial \xi^\alpha}{\partial x^\mu} \frac{\partial x^\mu}{\partial \sigma},
\]
where $\xi^\alpha$ represents the new inertial frame after coordinate change.\\

Again, by the equivalence principle,
\begin{align*}
\frac{\partial^2 \xi^\alpha}{\partial \sigma^2} 
&= \frac{\partial}{\partial \sigma}
\left(\frac{\partial \xi^\alpha}{\partial x^\mu} \frac{\partial x^\mu}{\partial \sigma}\right) \\\\
&= \frac{\partial \xi^\alpha}{\partial x^\mu} \frac{\partial^2 x^\mu}{\partial \sigma^2}
+ \frac{\partial^2 \xi^\alpha}{\partial \sigma \partial x^\mu} \frac{\partial x^\mu}{\partial \sigma}
= 0 \\
\intertext{Apply the chain rule on the second term, as indicated by the brackets:}
0 &= \frac{\partial \xi^\alpha}{\partial x^\mu} \frac{\partial^2 x^\mu}{\partial \sigma^2}
+ \frac{\partial^2 \xi^\alpha}{\underbracket{\partial x^\nu} \partial x^\mu} \frac{\partial x^\mu}{\partial \sigma}
\underbracket{\frac{\partial x^\nu}{\partial \sigma}}
\intertext{Then, multiply both side by the Jacobian $\partial x^\lambda/ \partial \xi^\alpha$} 
0 &= \underbracket{\frac{\partial x^\lambda}{\partial \xi^\alpha}  \frac{\partial \xi^\alpha}{\partial x^\mu}} \frac{\partial^2 x^\mu}{\partial \sigma^2}
+ \frac{\partial x^\lambda}{\partial \xi^\alpha} \frac{\partial^2 \xi^\alpha}{\partial x^\nu \partial x^\mu} \frac{\partial x^\mu}{\partial \sigma}
\frac{\partial x^\nu}{\partial \sigma}, 
\end{align*}
and recognize that the annotated contraction results in the Kronecker delta 
$\delta^{\lambda}_{\mu}$, a tensor equivalence of the identity matrix. By observing 
the structure of $\delta^{\lambda}_{\mu}$ (indexed component equals 1 when 
$\lambda=\mu$, and 0 otherwise), one shall realize its index-altering 
ability in Einstein summation notation [33]. For example:
\[
V^{\mu}\delta^{\lambda}_{\mu} = V^{\lambda}
\]
for an arbitrary vector $V^\mu$. Using such property, we yield the equation 
for the light beam's motion:
\begin{equation*}
0 = \frac{\partial^2 x^\lambda}{\partial \sigma^2} 
+ \frac{\partial x^\lambda}{\partial \xi^\alpha} 
\frac{\partial^2 \xi^\alpha}{\partial x^\nu \partial x^\mu} 
\frac{\partial x^\mu}{\partial \sigma}
\frac{\partial x^\nu}{\partial \sigma}
\tag{1}
\end{equation*}
which connects the ray's acceleration with its velocity with respect 
to coordinate time. Steven Weinberg's deduction in [32] ends here, 
followed by defining the factor in front of the second term as the 
affine connection (i.e., the Christoffel symbol in the main paper). 
However, we could proceed to show that this factor can by written entirely 
in terms of the metric tensor of coordinate system $\xi$, its derivative, and 
its inverse. \\

By the transformation rule of metric tensors (See Appendix 1), it can be shown 
that the metric tensor $g_{\alpha\beta}$ of the coordinate system of 
$\xi^\mu$ takes the form:
\[
g_{\alpha\beta} = \eta_{AB} \frac{\partial \xi^A}{\partial x^\alpha} \frac{\partial \xi^B}{\partial x^\beta},
\]
with its inverse
\[
g^{\alpha\beta} = \eta^{AB} \frac{\partial x^\alpha}{\partial \xi^A} \frac{\partial x^\beta}{\partial \xi^B}
\]
One can easily validate that contracting $g_{\alpha\beta}$ with 
$g_{\beta\gamma}$ by the above expansion indeed produces the 
Kronecker delta.\\\\
Now consider the derivative
\[
\frac{\partial}{\partial x^\alpha} g_{\mu\nu} = 
\frac{\partial}{\partial x^\alpha} \eta_{AB} \frac{\partial \xi^A}{\partial x^\mu} \frac{\partial \xi^B}{\partial x^\nu}
\]
By the product rule and the invariance of the Minkowski metric we obtain
\begin{align*}
\frac{\partial}{\partial x^\alpha} g_{\mu\nu} &= 
\eta_{AB} \left(\frac{\partial^2 \xi^A}{\partial x^\alpha \partial x^\mu} \frac{\partial \xi^B}{\partial x^\nu}
+ \frac{\partial \xi^A}{\partial x^\mu} \frac{\partial^2 \xi^B}{\partial x^\alpha \partial x^\nu}\right) \\
\intertext{The similar applies for the permutations of indices:} 
\frac{\partial}{\partial x^\nu} g_{\alpha\mu} &= 
\eta_{AB} \left(\frac{\partial^2 \xi^A}{\partial x^\nu \partial x^\alpha} \frac{\partial \xi^B}{\partial x^\mu}
+ \frac{\partial \xi^A}{\partial x^\alpha} \frac{\partial^2 \xi^B}{\partial x^\nu \partial x^\mu}\right) \\
\frac{\partial}{\partial x^\mu} g_{\nu\alpha} &= 
\eta_{AB} \left(\frac{\partial^2 \xi^A}{\partial x^\mu \partial x^\nu} \frac{\partial \xi^B}{\partial x^\alpha}
+ \frac{\partial \xi^A}{\partial x^\nu} \frac{\partial^2 \xi^B}{\partial x^\mu \partial x^\alpha}\right)
\end{align*}
By the symmetry of the Minkowski metric (i.e., $\eta_{AB} = \eta_{BA}$), and the symmetry of 
second derivatives (i.e., $\partial / (\partial x^\mu \partial x^\nu) =  \partial / 
(\partial x^\nu \partial x^\mu) $), we can always rewrite any of the above derivtives with any pairs of 
$\xi^A$ and $\xi^B$ swaped. For example, 
\begin{align*}
\frac{\partial}{\partial x^\nu} g_{\alpha\mu} &= 
\eta_{AB} \left(\frac{\partial^2 \xi^A}{\partial x^\nu \partial x^\alpha} \frac{\partial \xi^B}{\partial x^\mu}
+ \frac{\partial \xi^A}{\partial x^\alpha} \frac{\partial^2 \xi^B}{\partial x^\nu \partial x^\mu}\right) \\
&= \eta_{AB} \frac{\partial^2 \xi^A}{\partial x^\nu \partial x^\alpha} \frac{\partial \xi^B}{\partial x^\mu}
+ \eta_{AB}\frac{\partial \xi^A}{\partial x^\alpha} \frac{\partial^2 \xi^B}{\partial x^\nu \partial x^\mu} \\
\intertext{Since the order of summation does not matter for $A$ and $B$,}
\frac{\partial}{\partial x^\nu} g_{\alpha\mu}
&= \eta_{BA} \frac{\partial^2 \xi^B}{\partial x^\nu \partial x^\alpha} \frac{\partial \xi^A}{\partial x^\mu}
+ \eta_{AB}\frac{\partial \xi^A}{\partial x^\alpha} \frac{\partial^2 \xi^B}{\partial x^\nu \partial x^\mu} \\
&= \eta_{AB} \frac{\partial^2 \xi^B}{\partial x^\nu \partial x^\alpha} \frac{\partial \xi^A}{\partial x^\mu}
+ \eta_{AB}\frac{\partial \xi^A}{\partial x^\alpha} \frac{\partial^2 \xi^B}{\partial x^\nu \partial x^\mu} \\
&= \eta_{AB} \left(\frac{\partial^2 \xi^B}{\partial x^\nu \partial x^\alpha} \frac{\partial \xi^A}{\partial x^\mu}
+ \frac{\partial \xi^A}{\partial x^\alpha} \frac{\partial^2 \xi^B}{\partial x^\nu \partial x^\mu}\right)
\tag{2}
\end{align*}
At this point, one may conjecture that a connection exist between a
linear combination of the three derivatives and the factor in eq. (1). This is 
indeed correct. Consider the sum
\begin{align*}
\frac{\partial g_{\nu\alpha}}{\partial x^\mu}
+ \frac{\partial g_{\alpha\mu}}{\partial x^\nu}
- \frac{\partial g_{\mu\nu}}{\partial x^\alpha}
&= \eta_{AB} \Bigg[
    \left(
        \frac{\partial^2 \xi^A}{\partial x^\mu \partial x^\nu}
        \frac{\partial \xi^B}{\partial x^\alpha}
        +
        \frac{\partial \xi^A}{\partial x^\alpha}
        \frac{\partial^2 \xi^B}{\partial x^\nu \partial x^\mu}
    \right) \\
&\quad +
    \left(
        \frac{\partial^2 \xi^A}{\partial x^\nu \partial x^\alpha}
        \frac{\partial \xi^B}{\partial x^\mu}
        -
        \frac{\partial \xi^A}{\partial x^\mu}
        \frac{\partial^2 \xi^B}{\partial x^\alpha \partial x^\nu}
    \right) \\
&\quad +
    \left(
        \frac{\partial \xi^A}{\partial x^\nu}
        \frac{\partial^2 \xi^B}{\partial x^\mu \partial x^\alpha}
        -
        \frac{\partial^2 \xi^A}{\partial x^\alpha \partial x^\mu}
        \frac{\partial \xi^B}{\partial x^\nu}
    \right)
\Bigg]
\end{align*}
All grouped terms in $(\cdot)$ can be either combined or canceled  via manipulations 
like that of (2). Resulting in a simpler form:
\[
\frac{\partial g_{\nu\alpha}}{\partial x^\mu}
+ \frac{\partial g_{\alpha\mu}}{\partial x^\nu}
- \frac{\partial g_{\mu\nu}}{\partial x^\alpha}
=
2\eta_{AB}\frac{\partial^2 \xi^A}{\partial x^\mu \partial x^\nu}\frac{\partial \xi^B}{\partial x^\alpha}
\]
We see that this is already very close to the factor in (1). To continue the simplification, 
we contract the sum of derivatives with the inverse metric, specifically, summing over $\alpha$:
\begin{align*}
g^{\lambda\alpha}\left(\frac{\partial g_{\nu\alpha}}{\partial x^\mu}
+ \frac{\partial g_{\alpha\mu}}{\partial x^\nu}
- \frac{\partial g_{\mu\nu}}{\partial x^\alpha}\right)
&= 2 g^{\lambda\alpha} \eta_{AB}\frac{\partial^2 \xi^A}{\partial x^\mu \partial x^\nu}\frac{\partial \xi^B}{\partial x^\alpha} \\
&= 2 \eta^{CD} \frac{\partial x^\lambda}{\partial \xi^C} \underbracket{\frac{\partial x^\alpha}{\partial \xi^D}}
\eta_{AB}\frac{\partial^2 \xi^A}{\partial x^\mu \partial x^\nu} \underbracket{\frac{\partial \xi^B}{\partial x^\alpha}} \\
&= 2 \eta^{CD} \frac{\partial x^\lambda}{\partial \xi^C} 
\eta_{AB}\frac{\partial^2 \xi^A}{\partial x^\mu \partial x^\nu} \delta^{B}_{D} \\
&= 2 \eta^{CD} \delta^{B}_{D} \eta_{AB} \frac{\partial x^\lambda}{\partial \xi^C} 
\frac{\partial^2 \xi^A}{\partial x^\mu \partial x^\nu} \\
&= 2 \delta^{C}_{A}  \frac{\partial x^\lambda}{\partial \xi^C} 
\frac{\partial^2 \xi^A}{\partial x^\mu \partial x^\nu} \\
&= 2 \frac{\partial x^\lambda}{\partial \xi^A} 
\frac{\partial^2 \xi^A}{\partial x^\mu \partial x^\nu}
\end{align*}
Compare to (1), we see that this is exactly 2 times the factor in front of $\frac{\partial x^\mu}{\partial \sigma}
\frac{\partial x^\nu}{\partial \sigma}$. Thus we have proven that the null geodesic equation can 
be written as 
\[
\frac{\partial^2 x^\lambda}{\partial \sigma^2} + 
\frac{1}{2} g^{\lambda\alpha} \left( \frac{\partial g_{\nu\alpha}}{\partial x^\mu}
+\frac{\partial g_{\beta\mu}}{\partial x^\nu}
-\frac{\partial g_{\mu\nu}}{\partial x^\beta}\right)
\frac{\partial x^\mu}{\partial \sigma}
\frac{\partial x^\nu}{\partial \sigma}
 = 0
\]



% A template:      \frac{\partial}{\partial}


\end{document}